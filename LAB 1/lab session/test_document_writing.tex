\documentclass[12pt]{article}

\usepackage{graphicx}
\def\eq1{E=mc^2}
\begin{document}
	\title{ This is my first \LaTeX \ document}
	\author{Aparajita Dutta}
	\date{\today}
\maketitle
\tableofcontents
\newpage
\section{Lists}
	Tools for making sketches:
	
\begin{itemize}
	\item Pen
	\item Pencil
	\begin{itemize}
		\item Graphite
		\begin{itemize}
			\item 4B
			\item 8B
		\end{itemize}
		\item Charcoal
		\item Pastel
	\end{itemize}
	\item Paper
\end{itemize}

\begin{enumerate}
	\item Pen
	\item Pencil
	\begin{enumerate}
		\item Graphite
		\begin{enumerate}
			\item 4B
			\item 8B
		\end{enumerate}
		\item Charcoal
		\item Pastel
	\end{enumerate}
	\item Paper
\end{enumerate}

\section{equations}
\subsection{Inline equations}
The function is: $f(x) = x+1$

The second function is: $$f(y) = y+2$$

The third function is:
\begin{equation}
 f(y) = y-5
\end{equation}

Superscript and subscript: $f_x = x^{y-1}$

Fraction: $x = \frac{3}{4}$

Area of a circle: $\pi r^2$

Volume of a sphere: $(\frac{4}{3})\pi r^3$

\subsection{Array of equations}
Array of equation:
\begin{eqnarray}
f(x) = x+1\\
f(y) = y+1
\end{eqnarray}

\section{Brackets}
I have $\displaystyle\frac{2}{3}$ of a litre.

$a = \left\{\frac{b}{c}+c\right\}+d$
\\ \\
\section{Table}
\begin{table}[h]
	\centering
	\begin{tabular}{|l|c|r|}
	\hline
	$x$ & 1 & 2\\\hline
	$f(x)$ & 3 & 4\\\hline
\end{tabular}
\end{table}
\section{Graphics}
\begin{center}
\includegraphics[height=4cm, width=7cm]{images.png}
\end{center}

\section{Macros}
first use of \textsc{Einstein} equation \cite{ref1} is: $\eq1$

another use of Einstein equation \cite{ref1} is: $\eq1$

\begin{large} another use of Einstein equation is: $\eq1$ \end{large}

\begin{Large} another use of Einstein equation is: $\eq1$ \end{Large}

\begin{Huge} another use of Einstein equation is: $\eq1$ \end{Huge}

\begin{small} another use of Einstein equation is: $\eq1$ \end{small}

\begin{tiny} another use of Einstein equation is: $\eq1$ \end{tiny}

\bibliographystyle{plain}
\bibliography{testbib.bib}
%\begin{thebibliography}{9}
%	\bibitem{ref1}
%	D.~Uthor.
%	\newblock {\em D title}.
%	\newblock D Company, 2014.
%	
%	\bibitem{ref2}
%	D.~Uthor.
%	\newblock {\em D title}.
%	\newblock D Company, 2014.
%\end{thebibliography}
\end{document}